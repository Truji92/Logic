
\chapter*{Introducción}\addcontentsline{toc}{chapter}{Introducción}


	Hoy en día se generan grandes cantidades de datos con los que normalmente se trabaja de forma informatizada. En ocasiones trabajar sobre estos conjuntos puede ser complicado y costoso desde el punto de vista computacional, sobre todo al intentar aplicar mecanismos de aprendizaje automático o razonamiento como el Análisis formal de conceptos.
	
	Una forma de abordar este problema puede ser reducir el tamaño y complejidad de una base 
	de conocimiento antes de empezar a trabajar con ella, por ejemplo eliminando información no 
	relevante para lo que se intenta analizar. Sin embargo, esta eliminación no puede realizarse de forma arbitraria, debe hacerse respetando la consistencia lógica de los razonamientos que se realizan posteriormente.    
	
	Desde un punto de vista más humano, una persona es capaz de manejar simultáneamente varios
	ámbitos de información, siendo capaz de separarlos a la hora de tomar decisiones según la 
	situación en la que se encuentra. Por ejemplo, una persona modera su comportamiento en su hogar
	de forma diferente que en su entorno de trabajo, de forma que mientras trabaja tiene en cuenta factores
	a los que no atiende en su hogar aunque este tomando una decisión sobre el mismo asunto. Tomar un café 
	normal o descafeinado, puede tener diferentes respuestas en función del entorno o de la hora del día.
	
	El ser humano es capaz de razonar ignorando reglas que se aplican al problema que esta manejando en función del contexto. Podemos conseguir
	algo similar en sistemas basados en reglas mediante Retracción Conservativa.
	
	Este ``razonamiento bajo contexto'' puede aplicarse de forma computacional a las nuevas tecnologías. Por ejemplo, en un teléfono móvil con potencia limitada pueden utilizarse sensores (wifi, gps...) para aplicar filtros sobre un contexto y permitir ejecutar razonamiento para automatizar tareas trabajando sobre un sistema de tamaño reducido que el procesador del teléfono pueda manejar con soltura.
	
 
\section*{Objetivo}\addcontentsline{toc}{section}{Objetivo}

	El objetivo de este proyecto es la implementación de un sistema capaz de reducir y adaptar una base de conocimiento para su posterior uso en razonamiento bajo contexto. 
	
	Para ello se realizarán diferentes pasos:

	\begin{description}
	
		\item[Framework de lógica matemática]
		En primer lugar se implementará un framework que permita trabajar con entidades lógicas (fórmulas, implicaciones, clausulas, etc)
		que posteriormente facilitará la implementación y testeo del retractor.
		
		\item[Retractor de Implicaciones]
		Se realizará una implementación de un algoritmo retractor de implicaciones básico.
		
		\item[Optimizaciones]
		Por último se añadirán diferentes optimizaciones a la implementación básica del retractor.
	
	\end{description}

 