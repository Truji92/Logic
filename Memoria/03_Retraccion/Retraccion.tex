\chapter*{Retracción conservativa}\addcontentsline{toc}{chapter}{Retracción conservativa}

	
\section*{Retracción de teorías lógicas}\addcontentsline{toc}{section}{Retracción de teorías lógicas}
	
	En este apartado se presentan diferentes conceptos de lógica matemática necesarios para la correcta comprensión del resto de esta memoria.
	
	Aunque en nuestro caso el trabajo posterior se centra sobre el tratamiento de implicaciones, estas definiciones son genéricas y 
	son ciertas para la lógica matemática completa.
	
	
\subsection*{Extensión y Retracciones Conservativas}
	
	En terminos de lógica matemática se dice que una teoría $T$ es una \textbf{extensión conservativa} de una teoría $T'$ (o, lo que es lo mismo, $T'$ es una \textbf{retracción conservativa} de $T$) si toda consecuencia lógica de $T$ en el lenguaje de $T'$ es también consecuencia lógica de $T'$.
	
	Explicado de una forma más coloquial, esto significa que partiendo de una teoría escrita en un lenguaje podemos encontrar una teoría escrita en otro lenguaje más reducido. Todo lo que es cierto en esta nueva teoría también es cierto en la primera y todo lo que es cierto en la primera (y puede expresarse en el lenguaje de la segunda) es cierto. Esta nueva teoría será una retracción conservativa de la primera.
	
	Esto nos permite por ejemplo, poder demostrar algo en una teoría de tamaño más reducido y fácil de trabajar, sabiendo que el resultado será válido para cualquier extensión conservativa de esa teoría.
	
	
\subsection*{Aplicación de la retracción}
	
	Al trabajar en FCA podemos representar el conocimiento como conjuntos de implicaciones entre atributos y podemos utilizar los conceptos de extensión y retracción conservativa para facilitar el trabajo sobre esto conjuntos.
	
	Aplicar retracción de implicaciones al Análisis Formal de Conceptos nos permite obtener ese ``filtrado de conocimiento'' humano sobre una base de reglas, eliminando de la base todos los atributos que no tengan relevancia para nuestro estudio y por tanto reduciendo la complejidad del sistema pero a la vez manteniendo la validez de toda deducción o consecuencia lógica que obtengamos.
	
	
\section*{Retracción conservativa en lógica proposicional}\addcontentsline{toc}{section}{Retracción conservativa en lógica proposicional}

	A partir de esta definición de la retracción y extensión conservativa, en este apartado especificaremos como vamos a aplicar estos conceptos de forma concreta a la lógica proposicional ya que esta es la que se utiliza para trabajar con las implicaciones que obtenemos de FCA y razonar sobre ellas.

	Aplicando la definición de retracción de forma directa a la lógica proposicional podemos especificar algunos conceptos para optimizar la implementación. En este caso, el lenguaje es un conjunto de atributos y una teoría un conjunto de proposiciones que utilizan dichos atributos (a lo cual nos referimos como base de conocimiento).
	
	Aunque el concepto de proposición engloba diferentes tipos: negaciones, implicaciones, conjunciones, etc... Nosotros vamos a trabajar únicamente con implicaciones. Esto no afecta a los conceptos aquí explicados ya que nuestro trabajo sigue estando contenido dentro de la lógica proposicional.
	
	\begin{description}
		\item[Extensión y retracción conservativa] 
			Partiendo de un base de conocimiento $\mathcal{L}$ decimos que $\mathcal{L'}$ es una \textbf{extensión conservativa} 
			de $\mathcal{L}$ si $\mathcal{L'}$ contiene al menos todas los atributos presentes en $\mathcal{L}$ y 
			además todo lo que cierto en $\mathcal{L}$ lo es también en $\mathcal{L'}$. Del mismo modo podemos decir que $\mathcal{L}$ es una \textbf{retracción conservativa} de $\mathcal{L'}$. 
	\end{description}

	 Es decir, a partir de una base de conocimiento podríamos eliminar ciertos atributos de la misma y obtener una retracción conservativa, con menor número de atributos. Para esto definimos el \textbf{operador de olvido}. 
	 
	 \begin{description}
	 	\item[Operador de olvido]  	
	 	Un \textbf{operador de olvido} de un atributo $p$ es aquel que aplicado sobre dos proposiciones devuelve un 
	 	nuevo conjunto de proposiciones equivalentes pero que no contienen el atributo $p$.
	 	
	 	Si aplicamos este operador sobre una base de conocimiento
	 	nos permite obtener una nueva base de conocimiento que es una retracción conservativa de la anterior y no contiene el atributo $p$.
	 \end{description}
	 
	 Para explicar este operador, en primer lugar hablaremos de la interpretación polinomial de la lógica proposicional.
	 
	 \subsection*{Lógica proposicional y el anillo $\mathbb{F}_2[x]$}\addcontentsline{toc}{subsection}{Lógica proposicional y el anillo 			$\mathbb{F}_2[x]$}
	 	
	 	Existen teorías probadas que permiten transformar proposiciones lógicas a polinomios, permitiendo el uso de herramientas algebraicas para resolver problemas lógicos.
	 	
	 	Esta transformación da como resultado polinomios del anillo  $\mathbb{F}_2[x]$, es decir, polinomios de grado 1 cuyos coeficientes solo pueden ser 0 o 1.
	 	
	 	Las reglas utilizadas son:
	 	
	 	$\pi: Prop \to \mathbb{F}_2[x]$ para transformar proposiciones a polinomio, definida por las ecuaciones:
	 	
	 	\begin{itemize}
	 		\item $\pi(\bot) = 0$, $\pi(p_i) = x_i$, $\pi(\neg F) = 1 + \pi(F)$
	 		\item $\pi(F_1 \wedge F_2) = \pi(F_1) \cdot \pi(F_2)$
	 		\item $\pi(F_1) \vee F_2) = \pi(F_1) + \pi(F_2) + \pi(F_1) \cdot \pi(F_2)$
	 		\item $\pi(F_1 \to F_2) = 1 + \pi(F_1) + \pi(F_1) \cdot \pi(F_2)$
	 		\item $\pi(F_1 \leftrightarrow F_2) = 1 + \pi(F_1) + \pi(F_2)$
	 	\end{itemize}
 	
 		$\Theta: \mathbb{F}_2[x] \to Prop$ para transformar de polinomio a proposición lógica, definida por:
 		
 		\begin{itemize}
 			\item $\Theta(0) = \bot$, $\Theta(1) = \top$, $\Theta(x_1) = p_i$
 			\item $\Theta(a \cdot b) = \Theta(a) \wedge \Theta(b)$
 			\item $\Theta(a + b) = \neg(\Theta(a) \leftrightarrow \Theta(b))$
 		\end{itemize}
 	
 		Más información sobre estas transformaciones puede encontrarse en \cite{imp_polinomios}.
	 	
	 	A estos polinomios puede aplicarse cualquier operación algebraica, como por ejemplo la derivada. 
	 	
	 	Al derivar un polinomio en $\mathbb{F}_2[x]$ respecto a una variable dejando el resto constante, siempre obtendremos un nuevo polinomio que no contiene dicha variable (ya que si la variable aparece siempre tendrá exponente 1). Utilizando esta idea podemos buscar la forma de eliminar variables de nuestras fórmulas.
	 	
	 	Podemos definir una derivada lógica de la forma:
	 	
	 	$$
	 	\begin{array}{ccc}
	 	PForm                   & \stackrel{\partial}{\to}         & PForm          \\
	 	\pi \downarrow               &     \#        &  \uparrow  \Theta \\
	 	{\mathbb F}_2[{\bf x}]   & \stackrel{d}{\to}         & {\mathbb F}_2[{\bf x}]  \\ 
	 	\end{array}
	 	$$
	 	
	 	
	 	Siendo,
	 	
	 	\[
	 		\partial = \theta \circ d \circ \pi
	 	\]
	 	
	 	Utilizando esto, tal como se demuestra en \cite{calculemus}, puede definirse la \textbf{Regla de independencia} (o \textbf{regla $\partial$}) en fórmulas polinómicas que nos permite a partir de dos polinomios obtener uno nuevo que no contiene una variable concreta pero que es equivalente (desde el punto de vista lógico) a los iniciales. Esta regla puede utilizarse como un \textbf{operador de olvido} genérico.
	
	
 