\chapter*{Diseño e implementación}\addcontentsline{toc}{chapter}{Diseño e implementación}
	
	En este apartado se explican los detalles de la implementación realizada del programa así como su funcionamiento.
	
	El proceso de implementación se ha realizado en varias fases, en primer lugar se ha construido un pequeño framework o librería que permita trabajar con lógica matemática de cara a poder utilizar estas herramientas para facilitar la codificación del programa final y su testeo. Las funcionalidades contenidas en esta librería están pensadas para utilizarse como herramienta de desarrollo y por tanto un usuario no puede acceder a ellas de forma directa desde el programa final.
	
	Una vez construida esta base, se ha realizado la implementación del algoritmo de retracción así como varias optimizaciones sobre el mismo y finalmente se ha empaquetado como aplicación java de linea de comando en un archivo .jar.

\subsection*{Elección del lenguaje de programación}\addcontentsline{toc}{subsection}{Elección del lenguaje de programación}
	
	
	La implementación se ha realizado en Scala, un lenguaje orientado a objetos y funcional que se ejecuta sobre la JVM (Java Virtual Machine).
	
	Las razones por las que se ha escogido son:
	
	\begin{itemize}
		
		\item Funcional: Los lenguajes funcionales están planteados desde un punto vista muy cercano a las matemáticas por lo que implementar
		conceptos de lógica matemática en este tipo de lenguajes es más directo.
		
		\item JVM: Al ser un lenguaje que se ejecuta en el entorno Java, permite de forma fácil general un ejecutable que puede funcionar en cualquier plataforma.
		
	\end{itemize}




\section*{Framework de programación lógica}\addcontentsline{toc}{section}{Framework de programación lógica}

	Para implementar los diferentes conceptos de lógica de esta librería se ha tomado como base otra anterior, realizada en Haskell por \todo{referencia}, sobre la que se han realizado algunas modificaciones, mejoras y adaptaciones. Ya que Haskell es un lenguaje puramente funcional, el código se puede transformar a Scala de forma relativamente simple ya que aunque la sintaxis no es similar los conceptos con los que trabaja son los mismos.
	
	A continuación se describen los apartados más importantes de la librería y su funcionamiento.
	

\subsection*{Proposiciones}\addcontentsline{toc}{subsection}{Proposiciones}

	Los componentes más básicos de la librería son las estructuras de datos que permiten crear, almacenar y operar sobre proposiciones lógicas.
	
	Partiendo de un tipo \textit{Prop} (Proposición) y utilizando herencia y polimorfismo definimos siete tipos de datos con los que se puede construir cualquier proposición lógica:
	
	\begin{description}
		\item[Constante] Representa un valor booleano de verdadero o falso.
		\item[Atom] Representa una proposición que solo consta de una variable. Por ejemplo $p$ o $q$ serían representados mediante una instancia de tipo Atom.
		\item[Negación (Neg)] Representa la negación de otra proposición ($\neg$).
		\item[Conjunción (Conj)] Representa el operador lógico $\wedge$ (AND) entre dos proposiciones.
		\item[Disyunción (Disj)] Representa el operador lógico $\vee$ (OR) entre dos proposiciones.
		\item[Implicación (Impl)] Representa la implicación entre dos proposiciones. En lógica representado por $\rightarrow$.
		\item[Equivalencia (Equi)] Representa la equivalencia entre proposiciones. En lógica representado como $\leftrightarrow$.
	\end{description}

	Con estas siete estructuras puede representarse cualquier proposición lógica, por ejemplo:
	
	\[
		\neg p \vee (q \rightarrow r) = Disj(Neg(Atom(p)), Impl(Atom(q), Atom(r)))
	\]
	
	Tras la implementación de estas estructuras se realizó la implementación de un pequeño DSL (Domain specific language) que facilita la escritura de este tipo de fórmulas mediante el uso de algunas funciones y operadores. El ejemplo anterior escrito en este lenguaje quedaría: 
	
	\[
		no(q) \; OR \; (q \; -\!\!> \, r) 
	\]
	
	Otro concepto que necesita ser modelado es el de \textbf{Interpretación}, una interpretación es una asignación de valores verdadero o falso a las variables de una proposición, si tras la sustitución de los valores indicados por la interpretación la evaluación de la proposición es cierta, decimos que la interpretación es un \textbf{modelo} de dicha proposición.
	
	Esta entidad se maneja en la librería como un diccionario (un almacenamiento clave - valor) en el que se guardan las asignaciones a cada variable y la ausencia de alguna como clave se considera como que esa variable se asigna como falsa.
	
	Finalmente, se implementaron diferentes operaciones como obtener si una interpretación es modelo de una proposición o obtener todos los modelos existentes para una proposición, así como funciones que permiten operar directamente sobre conjuntos de proposiciones.
	
	\todo[inline]{Hasta que punto entrar en detalle en esta parte?}
	
	
\subsection*{Formas normales y cláusulas}\addcontentsline{toc}{subsection}{Formas normales y cláusulas}

	Una misma proposición puede representarse de diferentes formas, es usual en lógica trabajar sobre formas normalizadas de las proposiciones por lo que la librería permite la transformación de proposiciones a diferentes formas normales (forma normal negativa, conjuntiva y disyuntiva)	
	
	Una de las formas de representación más utilizadas es la \textbf{cláusula} y por ello se encuentra representada en la librería por un tipo de datos propio. Una cláusula es una proposición que solo contiene literales y disyunciones (o conjunciones aunque en nuestro caso este segundo modelo no se ha implementado), un \textbf{literal} puede ser un símbolo atómico (\textit{Atom} en la representación de proposiciones) o la negación de un símbolo atómico.
	
	De esta forma, modelando el tipo literal, podemos representar una clausula como un conjunto de literales, los cual es computacionalmente más fácil de manejar que la estructura recursiva que modela las proposiciones.
	
	Una de las operaciones más importantes que suele realizarse al trabajar con cláusulas es la de \textbf{resolvente}. Esta operación ya ha sido nombrada en la definición que dimos anteriormente para el operador de olvido para implicaciones. La resolvente de dos clausulas respecto a un literal es una nueva cláusula con la misma semántica que las anteriores pero que no contiene dicho literal. 
	
	La librería permite operaciones como el cálculo de resolventes de diferentes tipos así como cálculos de todas las resolventes posibles de conjuntos de clausulas, etc.
	

\subsection*{Razonamiento}\addcontentsline{toc}{subsection}{Razonamiento}

	Hasta ahora tenemos representación y operaciones sobre entidades lógicas pero la parte más interesante es poder realizar razonamiento lógico sobre estas representaciones.	
	
	La librería permite el cálculo diferentes conceptos como validez, consistencia, consecuencia lógica, etc... Mediante el uso de diferentes métodos que operan sobre las diferentes representaciones (proposiciones y cláusulas). Algunos de estos sistemas de razonamiento son:
	
	\begin{description}
		\item[Fuerza bruta] La librería contiene implementaciones exhaustivas de algunos de los conceptos mediante la comprobación de todas las posibilidades.
		\item[Tableros semánticos] Es un método que actúa sobre conjuntos de proposiciones para obtener todos los modelos del conjunto. Realizando modificaciones sobre el conjunto puede utilizarse para calcular diferentes conceptos, por ejemplo podemos probar que una proposición es un teorema si el conjunto de modelos de su negación es vacío, y este conjunto puede calcularse mediante tableros semánticos.
		\item[Davis-Putnam] Es un algoritmo que actúa sobre conjuntos de cláusulas para comprobar su satisfacibilidad, al igual que en el caso anterior pueden obtenerse diferentes resultados aplicándolo sobre modificaciones del conjunto inicial.
		\item[Cálculo de secuentes] Es un método de razonamiento que funciona sobre proposiciones pero que no solo nos permite probar fórmulas lógicas sino que nos indica el proceso de la prueba paso a paso, de forma que cada linea de la demostración utiliza las lineas anteriores de la misma.
	\end{description}

	Con estos métodos podemos aplicar razonamiento sobre cualquier conjunto de proposiciones o cláusulas que obtengamos, permitiéndonos por ejemplo comprobar si el conjunto de implicaciones generado al aplicar el algoritmo de retracción es equivalente al original.
	
	
\section*{Algoritmo retractor}\addcontentsline{toc}{section}{Algoritmo retractor}	
	
	Una vez construida la base de nuestra librería de lógica podemos empezar a implementar el algoritmo de retracción.
	
\subsection*{Retracción basada en polinomios}\addcontentsline{toc}{subsection}{Retracción basada en polinomios}

	La primera aproximación realizada al algoritmo de retracción se basa en una teoría que permite la transformación de proposiciones en polinomios sobre el anillo $\mathbb{F}_2[x]$. 
	
	\todo[inline]{
		referencia a calculemus y/o añadir explicación en profundidad de esto?
		\\ No tengo claro hasta que punto explicar esta parte ya que no hemos hablado de relacionarla con la otra retracción
	}	

	Una vez las proposiciones se convierten a polinomios puede trabajarse matemáticamente con ellos, aplicar un algoritmo de retracción basado en las derivadas de dichos polinomios y finalmente volver a convertir los polinomios a proposiciones.
	
	Este método es válido, capaz de funcionar con cualquier tipo de proposiciones pero presenta un problema. Durante el proceso de conversión de proposición a polinomio y viceversa, la expresión de la proposición sufre alteraciones (se forman estructuras anidadas complejas) y aunque el resultado obtenido es correcto desde el punto de vista lógico, las formulas son muy complejas desde el punto de vista humano por lo que el análisis a simple vista de los resultados se vuelve prácticamente imposible.
	
	
\subsection*{Retractor de implicaciones}\addcontentsline{toc}{subsection}{Retractor de implicaciones}
	
	La versión definitiva del algoritmo, tal como se ha mencionado anteriormente, funciona exclusivamente sobre implicaciones con la estructura utilizada en FCA.
	
	Para la implementación, en primer lugar se han utilizado unas versiones modificadas de las estructuras de datos \textit{Implicación} y \textit{Conjunción} de forma que estas no puedan contener otras proposiciones de forma recursiva:
	
	\begin{description}
		\item[Conjunción] En este caso, la conjunción en lugar de representar el operador binario $\wedge$ representa la conjunción sobre una lista de variables de tamaño indefinido.
		\item[Implicación] Una implicación se define como la implicación entre dos conjunciones.
	\end{description}

	Estas estructuras se han implementado de forma que su conversión a las estructuras proposicionales de la librería lógica es implícita y por tanto toda función de esta librería puede seguir aplicándose.
	
	Utilizando estas estructuras, se ha realizado una implementación del operador de olvido (ecuación \ref{operadorOlvido}) explicado en el apartado de retracción.
	
	Aplicando este operador sobre todas las combinaciones posibles de dos implicaciones de un conjunto podemos obtener uno nuevo que no contenga un atributo determinado. Sin embargo, debido a que el operador de olvido es simétrico las combinaciones con las mismas implicaciones y distinto orden son equivalentes, lo cual nos permite mejorar el orden de complejidad de este procedimiento.
	
	Utilizando este operador, se implementa el algoritmo de retracción siguiendo el siguiente pseudocódigo: 
	
	{
		\centering
		\noindent\fbox{
			\begin{minipage}{0.9\textwidth}
				\begin{algorithmic}[1]
					\Function{Retraccion}{$implicaciones, variable$}
						\State $nuevasImplicaciones \gets [\;]$
						\State $resto \gets implicaciones$
						\State
						\While{noVacío implicaciones}
							\State $actual \gets implicaciones.head$
							\State $nuevasImplicaciones \text{ ++ } implicaciones.map(
								 $
							\State $ \qquad impl \rightarrow opOlvido( impl, actual, variable )$
				
							\State $)$	
							
							\State $resto \gets implicaciones.tail$
						\EndWhile
						\State
						\State
						\Return$nuevasImplicaciones$
					\EndFunction
				\end{algorithmic}
			\end{minipage}
		}	

	}
	
	Tras su aplicación obtenemos una nueva lista de implicaciones que no contiene la variable indicada por parámetro (una \textbf{retracción conservativa} del conjunto).
	
	Este algoritmo tiene un orden temporal de ejecución de $O(n*\log{n})$ ya que es un bucle que recorre todas las implicaciones y en cada iteración recorre las implicaciones aún no recorridas por el bucle principal. 
	
	Sin embargo, la aplicación de manera sucesiva del algoritmo para eliminar varias variables no tiene por qué tener una tendencia a tardar menos tiempo ya que al aplicar el operador de olvido sobre dos implicaciones podemos generar varias (0, 1 ó 2) y estamos realizando esta operación de manera combinatoria sobre el conjunto completo. En definitiva, una aplicación de este algoritmo sobre un conjunto reduce en uno (o en cero si el conjunto no contenía previamente la variable) la cantidad de atributos diferentes pero puede aumentar la cantidad de implicaciones contenidas por el conjunto. 
	


\subsection*{Optimizaciones sobre el algoritmo}\addcontentsline{toc}{subsection}{Optimizaciones sobre el algoritmo}
	
	EL aumento del número de implicaciones mencionado en el apartado anterior es el factor que más afecta al rendimiento de ejecuciones sucesivas del algoritmo por lo que se han aplicado dos optimizaciones para intentar reducir sus efectos:

	\begin{description}
		\item[Implicaciones con consecuete vacío] 
			Una implicación con el 	consecuente (la parte a la derecha de la flecha) vacío no es útil a la hora de realizar razonamiento sobre un conjunto ya que no aporta ninguna información nueva. Por este motivo, todas las implicaciones que cumplan esta característica se filtran del conjunto final de implicaciones.
		\item[Implicaciones con el antecedente vacío] 
			Cuando una implicación no contiene ningún atributo a la izquierda (antecedente) significa que todos los atributos del consecuente pueden considerarse como ciertos sin ninguna precondición. 
	
			Basándonos en esto podemos sustituir todas la apariciones de esos atributos por el valor lógico \textit{cierto}. Y podemos ir más allá, puesto que tanto antecedente como consecuente de las implicaciones son conjunciones podemos aplicar que:
			
			\[
				\text{T} \wedge p \rightarrow p
			\]		
			
			Lo cual nos permite directamente eliminar todas las ocurrencias del atributo en cuestión en lugar de simplemente sustituirlas. 
			
			Esta optimización no produce de forma directa una reducción del número de implicaciones, pero su aplicación hace más probable que ocurran casos que permitan aplicar la optimización anterior al reducirse el número de atributos.	
	\end{description}


	Estas optimizaciones evidentemente no reducen el orden de complejidad del algoritmo pero reducen el número de reglas del conjunto, provocando de manera indirecta una reducción del tiempo cómputo de ejecuciones sucesivas del mismo.
	
\section*{Programa ejecutable}\addcontentsline{toc}{section}{Programa ejecutable}

	Con el objetivo de facilitar el uso del algoritmo su distribución se realiza empaquetándolo en un archivo ejecutable Java (.jar) de forma que puede ser utilizado en todo sistema compatible con Java y llamado desde cualquier lenguaje de programación o de forma manual desde un terminal de línea de comando. 	
	
	El ejecutable lee las implicaciones de un archivo de texto indicado por parámetro con el siguiente formato:
	
	\begin{itemize}
		\item Una implicación por línea
		\item Atributos separados por comas.
		\item Antecedente y consecuente de una implicación separados por el símbolo $=>$.
		\item No es necesario respetar ningún tipo de espaciado entre elementos.
	\end{itemize}

	Por ejemplo:
	
	\begin{lstlisting}
		
		n  =>  c, d 
		a, b, c, p  =>  g, r, t 
		d  =>  c 
		g  =>  c 
		r  =>      p, t, a,       b, g, c 
		a  =>  b, p 
		p, t   , a, d, b, g, c, r  =>  n 
		t  =>  p, a, b, g, c, r 
		b, c, g  =>  a, p, r, t 
	
	\end{lstlisting}
	
	El programa realiza una comprobación sobre la estructura de este archivo y es capaz de indicar el lugar en el que se ha cometido un error.
	
	A la hora de ejecutar el algoritmo, la aplicación permite algunas opciones:
	
	\begin{description}
		\item[Formato Otter] 
			Permite la opción de mostrar el resultado de la ejecución en formato Otter además del formato análogo al del fichero de entrada.
		\item[Traza] 
			El programa permite mostrar una traza de la ejecución del algoritmo en la que se muestra el proceso de eliminación de cada una de las variables elegidas así como la procedencia de cada implicación (indicando las implicaciones sobre las que se aplicó el operador de olvido para obtener la nueva implicación).
		\item[Tiempo de ejecución] 
			Permite la opción de mostrar el tiempo de ejecución del algoritmo tras terminar (sin contar el tiempo utilizado para el parseo de archivo u otras operaciones).
		\item[Sin optimizaciones] 
			Permite ejecutar el algoritmo sin las optimizaciones explicadas en el apartado anterior.
	\end{description}

	\todo[inline]{Anexo con una ejecución mostrando la traza??}
	
	Para facilitar la programación de la utilidad de línea de comando se ha utilizado una librería llamada \textit{scopt} \todo{Como se referencia esto?} que facilita la creación de un parser de opciones de línea de comando al estilo típico de Linux.
	
	Esta librería permite declarar de forma simple las opciones y su efecto y genera automáticamente documentación para el uso de la aplicación de línea de comando. La salida de la aplicación al cometer un error al llamarla o al utilizar la opción \textit{-help} es:


	\begin{verbatim}
	Implication Retractor 2.1
	Usage: Implication Retractor [options] <file>
	
	<file>              Fichero que contiene las fórmulas
	-v, --vars <value>  Lista de variables a eliminar
	-o, --otter         Mostrar salida en formato Otter
	-t, --trace         Mostrar traza de la ejecución
	-T, --timed         Mostrar el tiempo de ejecución del algoritmo
	--version1          Ultizar versión básica del algoritmo sin optimizaciones
	\end{verbatim}
	
	Y una llamada de ejemplo al mismo podría ser:
	
	\begin{verbatim}
	retractor.jar implicaciones.txt -v p,q,r --version1
	\end{verbatim}
	
	La cual generaría una retracción conservativa de las implicaciones contenidas en \textit{implicaciones.txt} sin los atributos \textit{p, q} y \textit{r} aplicando el algoritmo sin optimizaciones. 
	
	
		
		
	
	
	
	
	
	
	
	
	
	
	
	
	
	