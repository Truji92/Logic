
\chapter*{Conclusiones, limitaciones y trabajo futuro}\addcontentsline{toc}{chapter}{Conclusiones, limitaciones y trabajo futuro}

	La implementación realizada del retractor de implicaciones nos proporciona una prueba de concepto de que esta teoría puede aplicarse de forma satisfactoria a FCA (o a cualquier otro sistema que trabaje con implicaciones o proposiciones) y puede resultar útil a la hora de reducir la complejidad de los conjuntos o a eliminar variables que no nos interesan.
	
	Sin embargo nuestro programa está implementado de forma simple y a pesar de las optimizaciones realizadas no es eficiente a la hora de tratar con grandes cantidades de datos. Esto se debe a la naturaleza combinatoria del problema que añade una gran complejidad computacional.	
	
	Uno de los posibles trabajos futuros que derivan de forma inmediata, es realizar una implementación basada en computación distribuida (por ejemplo utilizando Apache Spark) que nos permita tratar de forma paralela el bucle principal del algoritmo en el que se realizan las combinaciones. Ya que cada iteración no depende de las anteriores es sencillo realizar una computación distribuida y finalmente unir los resultados.

	
	Por último, otra posibilidad de trabajo futuro es la refactorización de la librería de lógica junto con la construcción de una API pública para la misma, que permita su distribución como librería genérica de programación lógica en el entorno Java facilitando la creación de algoritmos y programas que trabajen con este tipo de datos.
	
	