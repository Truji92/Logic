\hypersetup{pdftitle={Retracción}} 

%%  Configuracion de paquetes
%%%%%%%%%%%%%%%%%%%%%%%%%%%%%%%%%%%%%%%%%%%%%%%%%%%%%%%%%%%%%%%%%%%%%
\renewcommand\lstlistingname{Listado}                   %  default is Listing
\renewcommand\lstlistlistingname{\'Indice de listados}  %  default is Listings 
%\renewcommand\thelstlisting{\thechapter .\arabic{lstlisting}} % captionstyle


\makeindex 

%*********************** FORMATO DE PÁGINA **************************

\voffset          -5mm
\cfoot{}

\pagestyle{fancy}
\fancyhead{} 
\fancyhead[RE]{{\small  \nouppercase{\leftmark}}}
%\fancyhead[RE]{{\small  \nouppercase{\today}}}
\fancyhead[LO]{{\small  \nouppercase{\rightmark}}}
\fancyhead[LE,RO]{\thepage}

\setlength{\headheight}{15pt}

\parskip 1.5ex 
%\setlength{\parskip}{2ex}              % despues del parrafo, doble linea

\itemsep= -5pt
%\topsep = -5pt
\setlength{\topsep}{-5pt}

\renewcommand{\baselinestretch}{1,2}

\voffset -0,2cm
\textheight 21.5cm
\textwidth 14.2cm

\oddsidemargin 1cm         % estos dos son para que no varíen tanto
\evensidemargin 0,5cm      % los margenes entre pags pares e impares

\let\headwidth\textwidth

\setcounter{tocdepth}{2}

\setlength{\marginparwidth}{27mm}


%%% Para los items de las listas
%%%%%%%%%%%%%%%%%%%%%%%%%%%%%%%%%%%%%%%%%%%%%%%%%%%%%%%%%%%%%%%%%%%%%
%\let\olditemize=\itemize
%\def\itemize{
%\olditemize
%\setlength{\itemsep}{-1ex}
%\setlength{\topsep}{-1ex}
%}
%\let\oldenumerate=\enumerate
%\def\enumerate{
%\oldenumerate
%\setlength{\itemsep}{-1ex}
%}








%%% Para que no aparezcan las cabeceras de las páginas que están en blanco
%%%%%%%%%%%%%%%%%%%%%%%%%%%%%%%%%%%%%%%%%%%%%%%%%%%%%%%%%%%%%%%%%%%%%
\makeatletter 
\def\cleardoublepage{\clearpage\if@twoside \ifodd\c@page\else
  \hbox{} 
  \thispagestyle{empty} 
  \newpage
  \if@twocolumn\hbox{}\newpage\fi\fi\fi} 
\makeatother


%%%  Cabecera nueva de los capítulos
%%%%%%%%%%%%%%%%%%%%%%%%%%%%%%%%%%%%%%%%%%%%%%%%%%%%%%%%%%%%%%%%%%%%%%
\makeatletter
\def\thickhrulefill{\leavevmode \leaders \hrule height 1ex \hfill \kern \z@}
\def\@makechapterhead#1{%
  %\vspace*{50\p@}%
  \vspace*{10\p@}%
  {\parindent \z@ \centering \reset@font
        \thickhrulefill\quad
        \scshape \@chapapp{} \thechapter
        \quad \thickhrulefill
        \par\nobreak
        \vspace*{10\p@}%
        \interlinepenalty\@M
        \hrule
        \vspace*{10\p@}%
        \Huge \bfseries #1\par\nobreak
        \par
        \vspace*{10\p@}%
        \hrule
    %\vskip 40\p@
    \vskip 100\p@
  }}
\def\@makeschapterhead#1{%
  %\vspace*{50\p@}%
  \vspace*{10\p@}%
  {\parindent \z@ \centering \reset@font
        \thickhrulefill
        \par\nobreak
        \vspace*{10\p@}%
        \interlinepenalty\@M
        \hrule
        \vspace*{10\p@}%
        \Huge \bfseries #1\par\nobreak
        \par
        \vspace*{10\p@}%
        \hrule
    %\vskip 40\p@
    \vskip 100\p@
  }}

%% Different font in captions
%%%%%%%%%%%%%%%%%%%%%%%%%%%%%%%%%%%%%%%%%%%%%%%%%%%%%%%%%%%%%%%%%%%%%
%\newcommand{\captionfonts}{\tiny}
%
%\makeatletter  % Allow the use of @ in command names
%\long\def\@makecaption#1#2{%
%  \vskip\abovecaptionskip
%  \sbox\@tempboxa{{\captionfonts #1: #2}}%
%  \ifdim \wd\@tempboxa >\hsize
%    {\captionfonts #1: #2\par}
%  \else
%    \hbox to\hsize{\hfil\box\@tempboxa\hfil}%
%  \fi
%  \vskip\belowcaptionskip}
%\makeatother   % Cancel the effect of \makeatletter

