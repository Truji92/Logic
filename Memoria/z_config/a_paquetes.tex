%
%%********************* PAQUETES  *******************
%

\usepackage[utf8]{inputenc}						% Para poder escribir acentos tal cual
\usepackage[spanish,activeacute]{babel}	% Titulos en español
\usepackage{fancyhdr}								% Estilo adornado
\usepackage{amsfonts}								% Para disponer de más símbolos matemáticos
\usepackage{amssymb}
\usepackage{amsmath}
\usepackage{hyperref}								% Para poner hiperenlaces
\usepackage[pdftex]{color}									% para importar dibujos coloreados
\usepackage{rotating}								% para usar \begin{sideways} que rota tabla 90 grados 
\usepackage{listings}								% para imprimir codigo fuente
\usepackage[colorinlistoftodos,textwidth=29mm,spanish]{todonotes} % para las notas de TODO:

\usepackage[all]{xy}
\usepackage{colortbl}								% Para tablas de colores
\usepackage{slashbox}								% Para poner las cabeceras partidas oblicuas
\usepackage{multirow}   % Para juntar columnas
\usepackage{multicol}
\usepackage{subcaption}

%%%  Paquetería necesaria de fábrica
%%%%%%%%%%%%%%%%%%%%%%%%%%%%%%%%%%%%%%%%%%%%%%%%%%%%%%%%%%%%%%%%%%%%%%
% \usepackage{palatino}        % Para el tipo de letra
%\usepackage{graphicx} % para importar combinados latex
%\usepackage{epsfig}          % para rotar figuras de Xfig  poniendo \begin{sideways} 
%\usepackage{stmaryrd}        % para usar la \bigsqcap
%\usepackage{cmtt}            % tipo de letra de las consolas.
%
%%% Paquetes de verbatim y extensión
%\usepackage{verbatim}        
%\usepackage{fancyvrb}
%\usepackage{shortvrb}
%\usepackage{amsmath}
%\usepackage{latexsym}
%
